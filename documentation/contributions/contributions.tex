\documentclass{sigchi}

% Use this section to set the ACM copyright statement (e.g. for
% preprints).  Consult the conference website for the camera-ready
% copyright statement.

% Use this command to override the default ACM copyright statement
% (e.g. for preprints).  Consult the conference website for the
% camera-ready copyright statement.

%% HOW TO OVERRIDE THE DEFAULT COPYRIGHT STRIP --
%% Please note you need to make sure the copy for your specific
%% license is used here!
% \toappear{
% Permission to make digital or hard copies of all or part of this work
% for personal or classroom use is granted without fee provided that
% copies are not made or distributed for profit or commercial advantage
% and that copies bear this notice and the full citation on the first
% page. Copyrights for components of this work owned by others than ACM
% must be honored. Abstracting with credit is permitted. To copy
% otherwise, or republish, to post on servers or to redistribute to
% lists, requires prior specific permission and/or a fee. Request
% permissions from \href{mailto:Permissions@acm.org}{Permissions@acm.org}. \\
% \emph{CHI '16},  May 07--12, 2016, San Jose, CA, USA \\
% ACM xxx-x-xxxx-xxxx-x/xx/xx\ldots \$15.00 \\
% DOI: \url{http://dx.doi.org/xx.xxxx/xxxxxxx.xxxxxxx}
% }

% Arabic page numbers for submission.  Remove this line to eliminate
% page numbers for the camera ready copy
% \pagenumbering{arabic}

% Load basic packages
\usepackage{balance}       % to better equalize the last page
\usepackage{graphics}      % for EPS, load graphicx instead 
\usepackage[T1]{fontenc}   % for umlauts and other diaeresis
\usepackage{txfonts}
\usepackage{mathptmx}
\usepackage[pdflang={en-US},pdftex]{hyperref}
\usepackage{color}
\usepackage{booktabs}
\usepackage{textcomp}
% Some optional stuff you might like/need.
\usepackage{microtype}        % Improved Tracking and Kerning
% \usepackage[all]{hypcap}    % Fixes bug in hyperref caption linking
\usepackage{ccicons}          % Cite your images correctly!
% \usepackage[utf8]{inputenc} % for a UTF8 editor only

% If you want to use todo notes, marginpars etc. during creation of
% your draft document, you have to enable the "chi_draft" option for
% the document class. To do this, change the very first line to:
% "\documentclass[chi_draft]{sigchi}". You can then place todo notes
% by using the "\todo{...}"  command. Make sure to disable the draft
% option again before submitting your final document.
\usepackage{todonotes}

% Paper metadata (use plain text, for PDF inclusion and later
% re-using, if desired).  Use \emtpyauthor when submitting for review
% so you remain anonymous.
\def\plaintitle{ExploX - Individual Contributions}
\def\plainauthor{Kevin M\"uller, Marc Rupp, Lukas Strobel, Xueting Li}
\def\emptyauthor{}
\def\plainkeywords{}

% llt: Define a global style for URLs, rather that the default one
\makeatletter
\def\url@leostyle{%
  \@ifundefined{selectfont}{
    \def\UrlFont{\sf}
  }{
    \def\UrlFont{\small\bf\ttfamily}
  }}
\makeatother
\urlstyle{leo}

% To make various LaTeX processors do the right thing with page size.
\def\pprw{8.5in}
\def\pprh{11in}
\special{papersize=\pprw,\pprh}
\setlength{\paperwidth}{\pprw}
\setlength{\paperheight}{\pprh}
\setlength{\pdfpagewidth}{\pprw}
\setlength{\pdfpageheight}{\pprh}

% Make sure hyperref comes last of your loaded packages, to give it a
% fighting chance of not being over-written, since its job is to
% redefine many LaTeX commands.
\definecolor{linkColor}{RGB}{6,125,233}
\hypersetup{%
  pdftitle={\plaintitle},
% Use \plainauthor for final version.
%  pdfauthor={\plainauthor},
  pdfauthor={\emptyauthor},
  pdfkeywords={\plainkeywords},
  pdfdisplaydoctitle=true, % For Accessibility
  bookmarksnumbered,
  pdfstartview={FitH},
  colorlinks,
  citecolor=black,
  filecolor=black,
  linkcolor=black,
  urlcolor=linkColor,
  breaklinks=true,
  hypertexnames=false
}

% create a shortcut to typeset table headings
% \newcommand\tabhead[1]{\small\textbf{#1}}

% End of preamble. Here it comes the document.
\begin{document}

\title{\plaintitle}

\numberofauthors{3}
\author{%
  \alignauthor{Kevin M\"uller\\
    \affaddr{Saarbr\"ucken, Germany}\\
    \email{s9kvmuel@stud.uni-saarland.de}}\\
  \alignauthor{Lukas Strobel\\
    \affaddr{St. Ingbert, Germany}\\
    \email{s8lustro@uni-saarland.de}}\\
  \alignauthor{Xueting Li\\
    \affaddr{Saarbr\"ucken, Germany}\\
    \email{ding14552@gmail.com}}\\
}

\maketitle

\section{Kevin M\"uller}
I am responsible for the complete implementation, where the main steps included
\begin{itemize}
\item setting up the server using Node.js.
\item implementing a secure user management as well as log-in via Strava (OAuth).
\item designing and implementing the database with tables for users, routes, activities and coordinates as well as spatial queries on geographic data.
\item doing front-end design and implementation using Jade as templating engine and Bootstrap.
\item integrating Leaflet.js with GeoJSON, MaskCanvas and Leaflet Routing Machine to display the activity map and routes. Preprocessing of the data in the backend.
\item implementating the relevant Strava API calls and corresponding database hooks to store and retrieve the data.
\item implementating a crawler to get all strava segments in a given area.
\item implementating the route generation algorithm as described in detail in the main paper.
\item implementating the user interface together with Lukas according to our UI prototypes for a good usability, especially for generating new routes.
\end{itemize}
... and of course some complementary work such as
\begin{itemize}
\item researching which libraries and services to use.
\item studying their documentations and APIs.
\item a lot of testing and debugging.
\item time management and issue tracking using the Github issue tracker.
\end{itemize}
... as well as project-related work including
\begin{itemize}
\item researching related work at the beginning of the seminar.
\item mostly doing implementation-related parts of the presentations and writing the implementation chapter of the report.
\item recording the video.
\item rather passively helping with the design of the study.
\end{itemize}

\section{Lukas Strobel}
I am responsible for Evaluation and Design of the UI.
For the Design part I...
\begin{itemize}
\item set up a Mockup Demo for further UI decisions
\item discussed with Kevin design choices he was implementing later
\end{itemize}
For the Evaluation part I...
\begin{itemize}
\item did research on how to test the software with users in the lab or in the field
\item communicated with the German triathlon squad at the Hermann-Neuberger-Sportschule
\item planned a field study the triathletes should be participating in
\item had trouble with their trainer because I could not rely on him
\end{itemize}
So after he did not answer back the third time after we had an agreement, I decided to switch to a lab study. Therefor I...
\begin{itemize}
\item planned a lab study with a think aloud observation, semi structured interview and the UEQ together with Xueting Li
\item gathered participants for the study from my training squad
\item contacted them and persuade them to participate
\item had seven athletes agreeing to support us with their participation in our study
\item created sample routes to fill our database with
\item invited them to come to my place
\item conducted seven iterations of think aloud observation, interview and UEQ with them (each of course one)
\item collected information (spoken words, interview answers) and put them all together in a document
\item did the written report of the interview
\end{itemize}

\section{Xueting Li}
TODO

\balance{}

\end{document}

%%% Local Variables:
%%% mode: latex
%%% TeX-master: t
%%% End:
